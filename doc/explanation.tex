\documentclass{article}

\usepackage[english]{babel}
\usepackage[latin1]{inputenc}
\usepackage[T1]{fontenc}
\usepackage{hyperref}
\usepackage[pdftex]{graphicx}

\newcommand{\comment}[1]{}

\begin{document}

\title{The GNUstep development installer}
\author{Tom Koelman}
\date{November 21, 2005}

\maketitle



This document describes what the Windows GNUstep development installer
does.

\section{Contents of the installer}

Basically it contains

\begin{itemize}
\item a shell
\item source
\item libraries
\item some scripts to compile the source
\item some patches on the source
\item some precompiled binaries if you decide not to compile the source
\end{itemize}

The shell is msys + mingw.

The source is

\begin{itemize}
\item ffcall
\item libobjc
\item base
\item gui
\item back
\item make
\end{itemize}

The libraries are

\begin{itemize}
\item jpeg
\item iconv
\item png
\item tiff
\item zlib
\end{itemize}


\section{Overview of running the installer}
It allows one to choose what it should do.

The choices are

\begin{description}
\item[Install all source plus binaries for base and gui] This would install everything to one directory, plus all already
compiled libraries of base and gui
\end{description}

\begin{description}
\item[Install all source and build binaries for base and gui] Installs everything to one directory, and then runs the scripts to
compile everything.
\end{description}

\begin{description}
\item[Install all source] Just installs everything and does nothing smart.
\end{description}


\section{Where is everything installed}
If you really want to figure out what happens, you could run the
``Install all source'' option and the look at the installed files. Say
you installed to \begin{quote}
\begin{verbatim}c:\\GNUstep\end{verbatim}
\end{quote}

Now all sources are installed in \begin{quote}
\begin{verbatim}c:\\GNUstep\\Development\\Source\end{verbatim}
\end{quote} The
sources are actually nothing more than the unpacked tar files.
The most interesting of the source directories is probably
\begin{quote}
\begin{verbatim}c:\\GNUstep\\Development\\Source\\patches\end{verbatim}
\end{quote} where you can find out what the
installer has patched for you.

Now take a look at \begin{quote}
\begin{verbatim}c:\\GNUstep\\Development\\msys\\1.0\\installer\end{verbatim}
\end{quote} This
contains all the scripts that would be run to compile the source. For
example, it contains the apply-patches.sh script that iterates over
all the patches in \begin{quote}
\begin{verbatim}c:\\GNUstep\\Development\\Source\\patches\end{verbatim}
\end{quote} and applies
them.
The subdirectory log here contains all the logging that was created during
the running of the installer. If you would actually have let the
installer compile for you, this one would contain interesting
logs. The temp directory here is used for miscellaneous purposes in
the scripts.


\section{Compiling}
All needed source is already untarred. Basically what the compile
scripts do is follow the recipe of the README.MinGW file. All
configuring and making and such is logged in
\begin{quote}
\begin{verbatim}c:\\GNUstep\\Development\\msys\\1.0\\installer\\log\end{verbatim}
\end{quote}


\section{Prebuilt binaries}
Actually, when the installer itself is built, first a version is
created that doesn't contain the binaries. Then it is run with the
option to compile everything from scratch. When that succeeds, a new
installer is built containing everything the original installer
contained plus the newly built binaries. In effect, the binaries that
are included are the ones in \begin{quote}
\begin{verbatim}c:\\GNUstep\\System\end{verbatim}
\end{quote}

After all is installed, the installer runs the
replace-gnustep-system-root.sh script. It changes the GNUstep.sh
script to contain the location of where the binaries are installed
instead of where they were originally compiled.



\end{document}
